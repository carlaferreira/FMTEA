\documentclass{llncs}

\usepackage{graphicx}
\usepackage{hyperref}
\renewcommand\UrlFont{\color{blue}\rmfamily}

\begin{document}
%
\title{Using TLA+ to teach broadcast protocols}

%\author{First Author\inst{1}\orcidID{0000-1111-2222-3333} \and
%Second Author\inst{2,3}\orcidID{1111-2222-3333-4444} \and
%Third Author\inst{3}\orcidID{2222--3333-4444-5555}}
%
%\authorrunning{F. Author et al.}
%
%\institute{Princeton University, Princeton NJ 08544, USA \and
%Springer Heidelberg, Tiergartenstr. 17, 69121 Heidelberg, Germany
%\email{lncs@springer.com}\\
%\url{http://www.springer.com/gp/computer-science/lncs} \and
%ABC Institute, Rupert-Karls-University Heidelberg, Heidelberg, Germany\\
%\email{\{abc,lncs\}@uni-heidelberg.de}}

\maketitle             

\begin{abstract}


\keywords{First keyword  \and Second keyword \and Another keyword.}
\end{abstract}

\section{Introduction}

\begin{itemize}
\item Inherent complexity of distributed systems
\item Fully understanding the details
\end{itemize}

\section{Broadcast protocols}

\begin{itemize}
\item Best-effort
\item Reliable 
\item Causal
\item Atomic
\end{itemize}


\section{Communication layer}

\begin{itemize}
\item easy to see the impact of different communication mechanisms (fifo, causal, unordered, etc)
\item  easy to change the underlaying assumptions (messages lost, messages corrupted, etc).
\end{itemize}

\section{Safety and liveness properties}

\subsection{Fairness properties}


\section{Conclusions}

\bibliographystyle{splncs04}
\bibliography{biblio}

\end{document}
